\chapter{Experimental Methods}
\ifpdf
    \graphicspath{{Experimental/ExperimentalFigs/PNG/}{Experimental/ExperimentalFigs/PDF/}{Experimental/ExperimentalFigs/}}
\else
    \graphicspath{{Experimental/ExperimentalFigs/EPS/}{Experimental/ExperimentalFigs/}}
\fi

%%%%%%%%%%%%%%%%%%%%%%%%%%%%%%%%%%%%%%%%%%%%%%%%%%%%%%%%%%%%

\section{2-bromo-2-(4-bromophenyl)acetic acid (\compound{cmpd:dibromoPAA})}
\begin{scheme}[ht]
    \schemeref[TMP1]{cmpd:4BrPAA}
    \schemeref[TMP2]{cmpd:dibromoPAA}
    \includegraphics{4-BrPAA-bromination.eps}
    \caption{Benzylic Bromination of \compound{cmpd:4-BrPAA}.\label{sch:BenzylBromination}}
\end{scheme}

\begin{experimental}[delta=(ppm),pos-number=sub,use-equal]
  \sisetup{separate-uncertainty,per-mode=symbol,detect-all,range-phrase=--}

A mixture of carboxylic acid \compound{cmpd:4BrPAA} (\SI{10.6}{\gram}, \SI{46.5}{\milli\mol}) and \ce{N-bromosuccinimide} (\SI{9.932}{\gram}, \SI{55.8}{\milli\mol}) were suspended in \ce{CCl4} (\SI{250}{\milli\L}) and the mixture was irradiated under refluxed with a \SI{240}{\watt} tungsten lamp.

An NMR of the crude mixture after three hours indicated incomplete conversion of the starting material. An additional 0.5 eq of NBS was addded and the reaction was again irradiated under reflux for a furthur 3 hours. The mixture was cooled, filtered, and the solvent along with elemental bromine were removed \invacuo. The residual orange solid was purified by flash column chromatography (2:1 Cyclohexane:Ethyl Acetate + \SI{1}{\percent} formic acid). Formic acid was coevaporated from the product with ethyl acetate to yield an off-white crystalline solid (\compound{cmpd:dibromoPAA}, \SI{9.733}{\gram}, \SI{71}{\percent}).

\NMR(500)[CDCl3] \val{5.31} (s, \#{1}, \pos{5}), \val{7.42--7.45} (m, \#{2}, \pos{11}), \val{7.50--7.53} (m, \#{2}, \pos{11}.
%
\NMR{13,C}(150)[CDCl3] \val{44.72} ($+$, \#{1}, \pos{1}), \val{23.4} ($+$,
\#{8}, \pos{5}), \val{126.0} ($+$, \#{4}, \pos{9}), \val{128.2} ($+$, \#{8},
\pos{3}), \val{130.8} ($+$, \#{2}, \pos{12}), \val{133.6} ($+$, \#{2},
\pos{11}), \val{137.0} ($+$, \#{4}, \pos{8}), \val{138.6} (q, \#{4},
\pos{2}), \val{140.6} (q, \#{2}, \pos{10}), \val{140.8} (q, \#{8}, \pos{4}),
\val{141.8} (q, \#{4}, \pos{6}), \val{145.6} (q, \#{2}, \pos{7})
%
  % \data{MS}[DCP, EI, \SI{60}{\electronvolt}] \val{703} (2, \ch{M+}), \val{582}
  % (1), \val{462} (1), \val{249} (13), \val{120} (41), \val{105} (100).
  % %
  % \data{MS}[\ch{MeOH + H2O + KI}, ESI, \SI{10}{\electronvolt}] \val{720} (100,
  % \ch{M+ + OH-}), \val{368} (\ch{M+ + 2 OH-}).
  % %
  % \data{IR}[KBr] \val{3443} (w), \val{3061} (w), \val{2957} (m), \val{2918}
  % (m), \val{2856} (w), \val{2729} (w), \val{1725} (w), \val{1606} (s),
  % \val{1592} (s), \val{1545} (w), \val{1446} (m), \val{1421} (m), \val{1402}
  % (m), \val{1357} (w), \val{1278} (w), \val{1238} (s), \val{1214} (s),
  % \val{1172} (s), \val{1154} (m), \val{1101} (w), \val{1030} (w), \val{979}
  % (m), \val{874} (m), \val{846} (s), \val{818} (w), \val{798} (m), \val{744}
  % (w), \val{724} (m), \val{663} (w), \val{586} (w), \val{562} (w), \val{515}
  % (w).
  %
\end{experimental}

%%%%%%%%%%%%%%%%%%%%%%%%%%%%%%%%%%%%%%%%%%%%%%%%%%%%%%%%%%%%

\section{2-bromo-2-(4-bromophenyl)-N-methoxy-N-methylacetamide (\compound{dibromoamide})}
To a stirred solution of \compound{cmpd:dibromoPAA} (\SI{4.409}{\gram}, \SI{15}{\milli\mol}) in dry DCM (\SI{120}{\milli\litre}) under argon at \SI{0}{\celsius} was added N,O-dimethylhydroxylamine hydrochloride (\SI{1.463}{\gram}, \SI{15}{\milli\mol}) followed by 1-(3-Dimethylaminopropyl)-3-ethylcarbodiimide hydrochloride (\SI{3.019}{\gram}, \SI{15.75}{\milli\mol}) in portions over \SI{15}{\minute}. Triethylamine (\SI{3.14}{\milli\litre}, \SI{22.5}{\milli\mol}) was then added and the mixture was allowed to rise to room temperature after stirring at \SI{0}{\celsius} for a further \SI{15}{\minute}.

After \SI{1}{\hour} the reaction mixture was washed with aq. \ce{HCl} (\SI{1}{\Molar}, 3x\SI{50}{\milli\litre}), sat. \ce{NaHCO3} (3x\SI{50}{\milli\litre}), brine (1x\SI{50}{\milli\litre}) and dried over \ce{MgSO4}. The solvent was removed \invacuo to yield \compound{dibromoamide} as a clear yellow oil which was not further purified.

%%%%%%%%%%%%%%%%%%%%%%%%%%%%%%%%%%%%%%%%%%%%%%%%%%%%%%%%%%%%

\section{S-(1-(4-bromophenyl)-2-(methoxy(methyl)amino)-2-oxoethyl) O-ethyl carbonodithioate (\compound{cmpd:bromoxanthateamide})}

A solution of \compound{dibromoamide} (\SI{4.586}{\gram}, \SI{13.6}{\milli\mol}) and potassium ethyl xanthate in acetone {\SI{100}{\milli\litre}} was  stirred at room temperature for \SI{3}{\hour} at which point TLC indicate complete consumption of the starting material. The solvent was remove \invacuo to yield a yellow oil which was taken up in water, extracted with \ce{DCM} (2x\SI{50}{\milli\litre}), the organics dried over \ce{MgSO4} and the solvent remove to yield a yellow oil which formed a crystalline solid on standing overnight. The solid was purified by column chromatography (2:1 cyclohexane:ethyl acetate) to yield \compound{cmpd:bromoxanthateamide} as a perlescent white crystaline solid (\SI{4.3523}{\gram}, \SI{11.5}{\milli\mol}, \SI{84.5}{\percent}) on standing overnight.

%%%%%%%%%%%%%%%%%%%%%%%%%%%%%%%%%%%%%%%%%%%%%%%%%%%%%%%%%%%%

\section{2-(4-bromophenyl)-2-mercapto-N-methoxy-N-methylacetamide (\compound{cmpd:bromothioamide})}

Piperidine (\SI{1.02}{\milli\litre}, \SI{10.2}{\milli\mol}) was added dropwise to a stirred mixture of \compound{cmpd:bromoxanthateamide} (\SI{3.4714}{\gram}, \SI{9.2}{\milli\mol}) in dry \ce{DCM} which had been deoxygenated by repeated argon/vacuum purges and placed under argon on a ice/water bath. TLC indicated that starting material remained after \SI{1}{\hour} at which point an additional 0.2 eq of \ch{piperidine} was added followed by another 0.4 eq in two portions.

The reaction was stopped with the addition of aq. \ce{HCl} (\SI{1}{\Molar}, \SI{50}{\milli\litre}), the organics were separated, washed with \ce{HCl} (\SI{1}{\Molar}, 2x\SI{50}{\milli\litre}, brine {\SI{50}{\milli\litre}), dried over \ce{MgSO4}, and the solvent removed \invacuo. The residue was purified by flash chromotography (2:1 \ce{cyclohexane}:\ce{ethyl acetate} + \SI{1}{\percent} formic acid) to yield starting material, a mixed fraction and an oil which crystalised as a white solid \compound{cmpd:bromothioamide} on cooling (\SI{1.083}{\gram}, \SI{3.82}{\milli\mol}, \SI{37.4}{\percent}).

%%%%%%%%%%%%%%%%%%%%%%%%%%%%%%%%%%%%%%%%%%%%%%%%%%%%%%%%%%%%

\section{2-(4-bromophenyl)-N-methoxy-N-methyl-2-(4,4'-dimethoxytritylthio)acetamide (\compound{cmpd:bromoDMTamide})}

To a stirred solution of \compound{cmpd:bromothioamide} (\SI{1.069}{\gram}, \SI{3.68}{\milli\mol}) in DCM was added \ce{4,4'-dimethoxytrityl chloride} (\SI{1.372}{\gram}, \SI{4.05}{\milli\mol}) followed by triethylamine (\SI{0.56}{\milli\litre}, \SI{4.05}{\milli\mol}). After \SI{1}{\hour} an additional 0.2 eq of \ce{4,4'-DMT.Cl} and triethylamine were added. The reaction was worked up after a further \SI{30}{\minute} by washing the organics with water (2x\SI{70}{\milli\litre}), drying over \ce{MgSO4}, and removing the solvent \invacuo. The crude was purified by flash column chromatography (gradiant from 6:1 to 2:1 \ce{cyclohexane}:\ce{ethyl acetate} + \SI{0.5}{\percent} \ce{dimethylethylamine}) to yield \compound{cmpd:bromoDMTamide} as a fine white crystalline solid (\SI{1.8419}{\gram}, \SI{3.1}{\milli\mol}, \SI{84}{\percent}).

%%%%%%%%%%%%%%%%%%%%%%%%%%%%%%%%%%%%%%%%%%%%%%%%%%%%%%%%%%%%

\section{2-(4-bromophenyl)-2-(4,4'-dimethoxytritylthio)acetaldehyde (\compound{cmpd:bromoDMTaldehyde})}

A solution of \ce{LiAlH4} (\SI{3}{\Molar} in \ce{THF}, \SI{220}{\micro\litre}) was added dropwise over the course of \SI{10}{minute} to a stirred solution of the protected amide \compound{cmpd:bromoDMTamide} (\SI{1.068}{\gram}, \SI{3.68}{\milli\mol}) in dry THF (\SI{30}{\milli\litre} under an argon atmosphere while cooling in a dry ice/\ce{isopropanol} bath.

After \SI{30}{\minute}, TLC indicated that all of the starting material had been consumed. A solution of \ce{NaHSO4} {\SI{1}{\Molar}} was added beginning dropwise at first to quench the reaction mixture followed by DCM (\SI{100}{\milli\litre}) and water (\SI{50}{\milli\litre} to improve phase separation. The organics were separated, washed with a further portion of \ce{NaHSO4} solution (\SI{1}{\Molar}, \SI{50}{\milli\litre}) and the combined aqueous washes were back extracted with DCM (\SI{20}{\milli\litre}), the organics dried over \ce{MgSO4}, and the solvent removed \invacuo to yield a yellow oil which was purified by flash column chromatography (4:1 \ce{cyclohexane}:\ce{ethyl acetate} + \SI{0.5}{\percent} \ce{triethylamine}) to yield \compound{cmpd:bromoDMTaldehyde} as a red crystalline solid (\SI{0.2112}{\gram}, \SI{0.40}{\milli\mol}, \SI{61}{\percent}).

% Better to dry over a neutral drying agent!

%%%%%%%%%%%%%%%%%%%%%%%%%%%%%%%%%%%%%%%%%%%%%%%%%%%%%%%%%%%%

\section{2-bromo-2-(4-methoxyphenyl)acetic acid}
2nd brominaton failed both
times

In the case of 4-methoxyphen
ylacetic acid, decarboxylation was obsevered with the product of a complex mix of products which were not possibly to separate by standard means.

%%%%%%%%%%%%%%%%%%%%%%%%%%%%%%%%%%%%%%%%%%%%%%%%%%%%%%%%%%%%

% ------------------------------------------------------------------------

%%% Local Variables:
%%% mode: latex
%%% TeX-master: "../thesis"
%%% End:
