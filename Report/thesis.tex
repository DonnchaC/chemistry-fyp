%input macros (i.e. write your own macros file called MacroFile1.tex)
%\include{Macros/MacroFile1}

\documentclass[oneside,12pt, report]{Classes/CUEDthesisPSnPDF}
%\documentclass[oneside,12pt]{report}

\ifpdf
    \pdfinfo { /Title  (CUED PhD and MPhil Thesis Classes)
               /Creator (TeX)
               /Producer (pdfTeX)
               /Author (Harish Bhanderi harish.bhanderi@cantab.net)
               /CreationDate (D:20030101000000)  %format D:YYYYMMDDhhmmss
               /ModDate (D:20030815213532)
               /Subject (Writing a PhD thesis in LaTeX)
               /Keywords (PhD, Thesis)}
    \pdfcatalog { /PageMode (/UseOutlines)
                  /OpenAction (fitbh)  }
\fi

\title{Writing a PhD Thesis\\[1ex]
        in \LaTeXe}

\ifpdf
  \author{\href{mailto:harish.bhanderi@cantab.net}{Harish Bhanderi}}
  \collegeordept{\href{http://www.eng.cam.ac.uk}{Department of Engineering}}
  \university{\href{http://www.cam.ac.uk}{University of Cambridge}}
% insert below the file name that contains the crest in-place of 'UnivShield'
  \crest{\includegraphics[width=30mm]{UnivShield}}
\else
  \author{Harish Bhanderi}
  \collegeordept{Department of Engineering}
  \university{University of Cambridge}
% insert below the file name that contains the crest in-place of 'UnivShield'
  \crest{\includegraphics[bb = 0 0 292 336, width=30mm]{UnivShield}}
\fi
%
% insert below the file name that contains the crest in-place of 'UnivShield'
% \crest{\IncludeGraphicsW{UnivShield}{40mm}{14 14 73 81}}
%
%\renewcommand{\submittedtext}{change the default text here if needed}
\degree{Doctor of Philosophy}
\degreedate{Yet to be decided}

% turn of those nasty overfull and underfull hboxes
\hbadness=10000
\hfuzz=50pt

% Put all the style files you want in the directory StyleFiles and usepackage like this:
% \usepackage{StyleFiles/watermark}
\usepackage[version=3]{mhchem}

% - use EPS with pdfLaTeX, essential for chemscheme ----
% \usepackage[runs=2]{auto-pst-pdf}
\usepackage{chemnum}
\usepackage[journal=rsc]{chemstyle}
\usepackage{bpchem,upgreek}
% \cmpdsetup{sub-input-sep=!}

\usepackage{chemmacros}

\usepackage{siunitx}

\usepackage[T1]{fontenc}
\usepackage{lmodern}
\usepackage{geometry} % Easy page layou
\usepackage[small,compact]{titlesec}

% \titleformat*{\section}{\LARGE\bfseries}
% \titleformat*{\subsection}{\Large\bfseries}
\titleformat{\subsection}
{\normalfont\large\bfseries}{\thesubsection}{0.8em}{}
% \titleformat*{\subsubsection}{\large\bfseries}
% \titleformat*{\paragraph}{\large\bfseries}
% \titleformat*{\subparagraph}{\large\bfseries}

% - optional command that make compound names easier to see ----
\renewcommand*{\schemerefformat}{%
  \color{magenta}\textit
}%

% - Don't format titles in PDF outline
\pdfstringdefDisableCommands{%
  \let\IUPAC\@firstofone
  \let\|\relax
}

\doublespacing

\begin{document}

%\language{english}

% A page with the abstract on including title and author etc may be
% required to be handed in separately. If this is not so, then comment
% the below 3 lines (between '\begin{abstractseparte}' and
% 'end{abstractseparate}'), normally like a declaration ... needs some more
% work, mind as environment abstracts creates a new page!
% \begin{abstractseparate}
%   
% Thesis Abstract -----------------------------------------------------


%\begin{abstractslong}    %uncommenting this line, gives a different abstract heading
\begin{abstracts}        %this creates the heading for the abstract page

The synthesis of two ring substituted 2-mercapto-2-phenylethyl acyl transfer auxiliaries is reported. We have demonstrated their utility in extended chemical ligation by assembling a 14 mer model peptide via cysteine-free auxiliary mediated ligation at a Gly-Gly ligation site. Finally we have tuned the auxiliary cleavage conditions to form the desired native peptide in excellent yields.

\end{abstracts}
%\end{abstractlongs}


% ----------------------------------------------------------------------


%%% Local Variables:
%%% mode: latex
%%% TeX-master: "../thesis"
%%% End:

% \end{abstractseparate}




% Using the watermark package which is in StyleFiles/
% and to remove DRAFT COPY ONLY appearing on the top of all pages comment out below line
%\watermark{DRAFT COPY ONLY}


\maketitle

%set the number of sectioning levels that get number and appear in the contents
\setcounter{secnumdepth}{3}
\setcounter{tocdepth}{3}

% \frontmatter % book mode onlyd
\pagenumbering{roman}
% Thesis Acknowledgements ------------------------------------------------


%\begin{acknowledgementslong} %uncommenting this line, gives a different acknowledgements heading
\begin{acknowledgements}      %this creates the heading for the acknowlegments

I would like to thank Prof. Oliver Seitz and the Department of Chemistry, Humboldt University Berlin for providing me the opportunity to complete this research in their laboratories. I thank Simon Loibl and the research group for their invaluable guidance and support while undertaking this work.


\end{acknowledgements}
%\end{acknowledgmentslong}

% ------------------------------------------------------------------------

%%% Local Variables:
%%% mode: latex
%%% TeX-master: "../thesis"
%%% End:


% Thesis Abstract -----------------------------------------------------


%\begin{abstractslong}    %uncommenting this line, gives a different abstract heading
\begin{abstracts}        %this creates the heading for the abstract page

The synthesis of two ring substituted 2-mercapto-2-phenylethyl acyl transfer auxiliaries is reported. We have demonstrated their utility in extended chemical ligation by assembling a 14 mer model peptide via cysteine-free auxiliary mediated ligation at a Gly-Gly ligation site. Finally we have tuned the auxiliary cleavage conditions to form the desired native peptide in excellent yields.

\end{abstracts}
%\end{abstractlongs}


% ----------------------------------------------------------------------


%%% Local Variables:
%%% mode: latex
%%% TeX-master: "../thesis"
%%% End:


\tableofcontents
\listoffigures
\listofschemes
\printnomenclature  %% Print the nomenclature
\addcontentsline{toc}{chapter}{Nomenclature}

% \mainmatter % book mode only
%%% Thesis Introduction --------------------------------------------------
\chapter{Introduction}
\ifpdf
    \graphicspath{{Introduction/IntroductionFigs/PNG/}{Introduction/IntroductionFigs/PDF/}{Introduction/IntroductionFigs/}}
\else
    \graphicspath{{Introduction/IntroductionFigs/EPS/}{Introduction/IntroductionFigs/}}
\fi

Importance of proteins and petides. The ability to synthetically produce peptides provides fantastic tools to probe biological process and functions by cemical means.

The development of solid phase peptide synthesis provided the key technique for the sequential sythesis of proteins. Unfortunatly SPPS encouters difficult  when producing longer peptides due to aggregation factors. The sequential loses during a linear synthesis also lead to poor yields in longer sequences or in peptides with so called 'difficult sequeneces'.

\section{Peptide Ligation}

early approaches

\section{Native Chemical Ligation}

Capture followed by S->A acyl shift. First reported by '...'

Peptide ligation techniques help to overcome some of shortcommings of solid phase peptide synthesis. Allows for convergent synthesis of longer chain peptiddes or proteins by ligating small segments which are producable via SPPS.

Particularly useful for synthesis of long, and cyclic peptides
Allows synthesis of challenging sequences not possible by SPPS

site some examples of successes from NCL.
extended native chemical ligation

\subsection{Limitations of Native Chemical Ligation}

The requirment for an N-terminal cystine in standard Native Chemical Ligation is limitation. Cysteine has relativily low abundance in proteins (ref).

In some cases it is possible to substitue a cysteine for a (glycine/alanine) to produce a non-native peptide which maintains its natural function.

\section{ Native Chemical Ligation at Non-Cysteine Sites}

A number of approaches developed to extend the utility of NCL

    \subsection{Desulfurization Approaches}

    Synthesis of peptides conta

    \subsection{Auxiliary Mediated NCL}


    scheme for auxiliary mediated ligation

    Limited range of possible junctions,
    Issues with auxiliary removal

    Previously developed auxiliaries.
    3,4,5-trimethoxy and others.
    table of their yields/cleavage

    photolabile auxilaries

    current work focuses on a new set of auxiliaries

    2-phenyl-2-mercapto auxiliaries. This class of auxiliries have shown to be removable in the presence of TCEP. Previous work has indicated the auxiliary removals occurs by way of a radical mechanism.

    In this work two ring substituted 2-phenyl-2-mercapto auxiliaries are synthesised to probe robe substituent effects on both ligation efficieny but primarily on radical promoted auxiliary cleavage.


        \subsubsection{Auxillary Removal}

        The mechanism of removal for this class of auxiliaries is currently unknown. This work studies a the cleave of a 2-(4-bromophenyl)-2-mercapto system. It is hoped that presence of bromine in the auxiliary should allow for the identificasistion of auxiliary cleavage products from the post cleavage crude by mass spectrum/MALDI analysis.

        The synthesising removal for this cl bromoauxiliary was choosen to probe the relative influence of inductive and resonance effects on the ring and the stabilisation of the hypothesised radical intermediate

        site, TCEP radical sulfhydral cleavage.

%%% ----------------------------------------------------------------------


%%% Local Variables:
%%% mode: latex
%%% TeX-master: "../thesis"
%%% End:

% \pagebreak[4]
% \hspace*{1cm}
% \pagebreak[4]
% \hspace*{1cm}
% \pagebreak[4]

\chapter{Results and Discussion}
\ifpdf
    \graphicspath{{Results/ResultsFigs/PNG/}{Results/ResultsFigs/PDF/}{Results/ResultsFigs/}}
\else
    \graphicspath{{Results/ResultsFigs/EPS/}{Results/ResultsFigs/}}
\fi

\cmpd*{cmpd:bromoacid}
\cmpd*{cmpd:meoamide}
\cmpd*{cmpd:bromoamide.{one,two}}
\cmpd*{cmpd:xanthateoamide.{one,two}}
\cmpd*{cmpd:thioamide.{one,two}}
\cmpd*{cmpd:DMTamide.{one,two}}
\cmpd*{cmpd:DMTaldehyde.{one,two}}

The experiments conducted and results are set out and described. Results should be discussed and set in context with the recent literature so that their significance is evident. A clear story should be developed so that the reader is lead through the project, understanding why experiments were performed and the  relevance of the results at each stage. The R\&D section should lead on from the introduction, and into the conclusion, building on ideas from the introduction and clearly highlighting key results for the conclusion.

\section{Synthesis of 2-mercapto-2-phenylacetaldehyde auxiliaries \cmpd+{cmpd:DMTaldehdye}}

  \begin{scheme}[H]
      \cmpdref{cmpd:DMTaldehyde.one}
      \cmpdref{cmpd:DMTaldehyde.two}
      \includegraphics[width=\textwidth]{twoauxiliaries.eps}
      \caption{2-mercapto-2-phenylacetaldhyde ligation auxiliaries.\label{sch:TwoAuxiliaies}}
  \end{scheme}

  As part of this project we synthesized a set of two substituted S-protected 2-mercapto-2-phenylacetaldehyde auxiliaries \cmpd+{cmpd:DMTaldehdye.one, cmpd:DMTaldehyde.two} starting from the commerical available substituted phenylacetic acids via a 6-step route (\ref{sch:OverallScheme}) with good yields overall.

  Both auxiliaires were synthesised bearing an acid-labile 4,4'-dimethoxytrityl (DMT) thiol protecting group to prevent thioacetal formation. The DMT protecting group was selected as it is rapidly removed  under the TFA conditions employed when cleaving the auxiliary peptides from the solid support.

  \begin{scheme}[H]
      \cmpdref{cmpd:bromoacid}
      \cmpdref{cmpd:meoamide}
      \cmpdref{cmpd:bromoamide.{one,two}}
      \cmpdref{cmpd:xanthateoamide.{one,two}}
      \cmpdref{cmpd:thioamide.{one,two}}
      \cmpdref{cmpd:DMTamide.{one,two}}
      \cmpdref{cmpd:DMTaldehyde.{one,two}}
      \includegraphics[width=\textwidth]{overall-scheme.eps}
      \caption{Overall route for auxiliary synthesis.\label{sch:OverallScheme}}
  \end{scheme}

  The synthesis of the 4-bromophenyl auxiliary \cmpd{cmpd:DMTaldehyde.one} proceeded without difficult via benzylic bromination of 4-bromophenylacetic acid to \cmpd+{cmpd:bromoacid}. Attempts to synthesize 2-bromo-2-(4-methoxyphenyl)acetic acid under the same reaction conditions failed with the formation of a complex mixture of products. Efforts to carry out the radical bromination of 4-methoxyphenylacetic acid under milder conditions (NBS, AIBN, \SI{30}{\celsius}) also yielded a complex mixture of products which were not readily purifiable by flash chromatography.

  A literature review indicated that 4-methoxyphenylacetic acid can undergo rapid decarboxylation to produce a benzylic radical after the formation of an intermediate aromatic radical cation [decarboxylation]. The route to \cmpd{cmpd:bromoamide.two} was modified by forming the Weinreb amide \cmpd{cmpd:meoamide} first which does cannot undergo this decarboxylation process. A subsequent AIBN promoted benzylic bromination on this substrate proceeded cleanly to afford \cmpd+{cmpd:bromoamide.two} which was used without further purifcation. The formation of the Weinreb amide before radical bromination is likely to provide the most consistant results when synthesising future substituted 2-mercapto-2-phenyl auxilaries.
  % TODO: NMR proof ^ ^

  The subsequent steps to synthesise the protected auxillaries \cmpd+{cmpd:DMTaldehyde.one, cmpd:DMTaldehyde.two} proceed smoothly and both were obtained as red crystalline solids in good yields overall (\cmpd+{cmpd:DMTaldehyde.one}=\SI{12}{\percent}, \cmpd+{cmpd:DMTaldehyde.two}=\SI{18}{\percent}).

  \begin{figure}[H]
    \centering
    \includegraphics{br-absorbance.eps}
    \caption{Absorbance of 4-bromophenyl auxiliary amide \cmpdref{cmpd:thioamide.one}}
    \label{fig:bramideabsorbance}
  \end{figure}

\section{Synthesis of model peptide}

In this work, the peptide \ce{HGRAEYSGLG-NH2} was choosen as a model peptide for examining the auxiliary promoted ligation at a non N-terminal cysteine junction. The peptide was synthesised using a standard Fmoc SPPS strategy using Tentagel Rink amide resin on [PEPTIDE SYNTHESISEr]. HCTU/Oxyma, DOUBLE COUPLING, CAPPING, DEPROTECTION.

\section{Coupling of auxiliaries to model peptide}

  \begin{scheme}[H]
      \cmpdref{cmpd:DMTaldehyde.{one,two}}
      \cmpdref{cmpd:auxiliary-peptide.{one,two}}
      \includegraphics[width=\textwidth]{reductive-amination.eps}
      \caption{Reductive amination of auxiliary to peptide.\label{sch:reductiveamination}}
  \end{scheme}

  The auxiliary aldehydes \cmpd+{cmpd:DMTaldehyde.one, cmpd:DMTaldehyde.two} were coupled to the N-terminal of the model peptide under standard reductive amination conditions as shown in \ref{sch:reductiveamination}. A solution of respective auxiliary and\ch{NaBH3CN} in NMP:IPA:AcOH (3:1:\SI{5}{\percent}) was shaken with the model peptide on solid support for \SI{12}{\hour}. The peptide auxiliaries were subsequently cleaved from the resin with TFA/TIS, precipitated from cold ether and purified by preparative HPLC.

  UPLC plot showing pure auxillary-peptides

  \subsection{Ligation of auxiliary peptides and thioester}

    \begin{scheme}[H]
        \cmpdref{cmpd:auxiliarypeptide.{one,two}}
        \cmpdref{cmpd:auxiliarydipeptide.{one,two}}
        \includegraphics[width=\textwidth]{auxiliary-ligation.eps}
        \caption{Native Chemical Ligation of auxiliary-peptide and thioester.\label{sch:auxiliaryncl}}
    \end{scheme}

    The auxiliary peptides \cmpd{cmpd:auxiliarypeptide.{one,two}} were combined with the model LYRAG peptide thioester \cmpd+{cmpd:thioester} and lyopholised. A ligation buffer (\SI{100}{\milli\Molar} \ch{Na2HPO4}, \SI{100}{\milli\Molar} TCEP, \SI{3}{\percent} thiophenol, adjusted to pH 7.5 with NaOH) was prepared with degassed millipore. TCEP acts to reduce any formed disulfide bonds. Thiophenol promotes the transthioesterification the the LYRAG alkyl thioester to a more reactive aryl thioester.
    [http://pubs.acs.org.elib.tcd.ie/doi/abs/10.1021/ja962656r]
    Acetonitrile (\SI{10}{\percent} was added to ensure complete dissolution of both peptides in the ligation buffer.

    The ligation was followed was measured by analysing aliquots of the ligation mixture by UPLC.

    The ligation progress was monitored by RP-HPLC analysis of aliquots of the reaction mixture over time. The ligation was observed to occur rapidly with comple consumption of the auxiliay peptide \cmpd+{cmpd:auxiliarypeptide.one} observed at the \SI{2.5}{\hour} point.

    [CONFIRM WITH HPLC graph].

    The ligation product \cmpd{cmpd:auxiliarydipeptide.one} was purified by preparative scale RP-HPLC to yield the isolated ligated auxiliary peptide in \SI{80}{\percent} yield.

    [CONFIRM WITH MS DATA AND HPLC CHART]

    compare these results with previous work, good yields

  \subsection{Auxiliary cleavage from ligated peptide}

    The final aim of this research was to explore the conditions affecting the radical induced cleavage of the auxiliary to form the desired native peptide. Previous work in this group has indicated that the cleavage occurs by a radical process but the exact mechanism is currently unknown.

    \subsection{Cleavage Condtions}

    To explore the conditions, both ligated auxiliary peptides were subjected to the cleavage conditions at varied concentrations and temperatures.

    TABLE of reactions for 4-bromo

    The 4-bromophenyl auxiliary \cmpd{cmpd:auxiliarydipeptide.one} showed an accelerated rate of cleavage and was almost complete removal was observed  after \SI{4}{\hour} (\SI{100}{\milli\molar}, \SI{50}{\celsius}). HPLC analysis of the products showed the formation of considerable byproducts via scission of the peptide amide backbone. The cleavage was repeated at varying temperatures and TCEP concentrations in an effort to minimise the observed peptide degredation.

    Graph of rates of cleavage for varying conditions

    TODO: SCHEME FOR AUXILARY CLEAVAGE TO GIVFREE DIPEPTIDE

    Describe the cleavage experiments performed. differen

    Theory about acceleated cleavage rate in 4-bromophenyl aux. Inductive effects with stabilize the the formation of a benzyl radical.

    \subsubsection{Cleavage Mechanism}

    A number of high resolution MALDI experiments were performed on the \cmpd{cmpd:auxiliarydipeptide.one} post cleavage reaction mixture. It was hoped that the characteristic bromine isotope twin MS peaks would allow for  auxiliary fragments to be identified from the complex post cleavage reaction mixture.

    A characteristic twin peak signal of the correct mass difference were not observered in the initial MALDI experiment. It is possible that the formed auxiliary fragments may not be readily protonated and in turn wouldn't be accelerated into the mass spectrometer. It may also be a case that formed  auxiliary fragments undergo radical self-polymerisation. These dimers or polymers would instead display a 1:2:1 type mass spectral pattern.

\begin{table}[h]
\centering
\begin{tabular}{@{}ccccccc@{}}
\toprule
\multirow{2}{*}{Concentration (mM)} & \multirow{2}{*}{Temperature (C)} & \multicolumn{5}{c}{Conversion (\%)} \\ \cmidrule(l){3-7}
                                    &                                  & 1h    & 2h   & 4h   & 6h    & 24h   \\ \cmidrule(r){1-2}
100                                 & 50                               & 55    & 82   & 99   & 100   &       \\
20                                  & 50                               &       &      &      &       &       \\
100                                 & 40                               &       &      &      &       &       \\
400                                 & RT                               &       &      &      &       &       \\ \bottomrule
\end{tabular}
\caption{Cleavage of 4-bromophenyl auxiliary}
\label{bromoauxiliarycleavagedata}
\end{table}

    explore mechanism of auxiliary cleavage based on observed products

% ------------------------------------------------------------------------


%%% Local Variables:
%%% mode: latex
%%% TeX-master: "../thesis"
%%% End:

\def\baselinestretch{1}
\chapter{Conclusions and Further Work}
\ifpdf
    \graphicspath{{Conclusions/ConclusionsFigs/PNG/}{Conclusions/ConclusionsFigs/PDF/}{Conclusions/ConclusionsFigs/}}
\else
    \graphicspath{{Conclusions/ConclusionsFigs/EPS/}{Conclusions/ConclusionsFigs/}}
\fi

\def\baselinestretch{1.66}

Further explorer auxiliaries with more electron donating para substituents and determine effect of substituents on rate.

%%% ----------------------------------------------------------------------

% ------------------------------------------------------------------------

%%% Local Variables:
%%% mode: latex
%%% TeX-master: "../thesis"
%%% End:

\chapter{Experimental Methods}
\ifpdf
    \graphicspath{{Experimental/ExperimentalFigs/PNG/}{Experimental/ExperimentalFigs/PDF/}{Experimental/ExperimentalFigs/}}
\else
    \graphicspath{{Experimental/ExperimentalFigs/EPS/}{Experimental/ExperimentalFigs/}}
\fi

%-----------------------------------------------------------

\section*{Materials and Methods}

All commercial material (Acros, Fluka, Sigma Aldrich) were used without purification. All solvents were reagent grade or HPLC grade. All reactions were performed under an atmosphere of argon. NMR spectra (\textsuperscript{1}H and \textsuperscript{13}C) were recorded on a \textit{Bruker Advance II 300} or\textit{Advance II 500}, referenced to TMS or residual solvent. Analytical TLC was performed on E. Merck silica gel 60 F254 plates and flash column chromatography was performed on silica gel 60 from \textit{Macherey Nagel}.

Preparative HPLC separations were performed on an \textit{Agilent 1100 Series} HPLC equipped with a Nucleodur C18 Gravity (\SI{5}{\micro\meter}) and Nucleosil 300-7 C4 columns from \textit{Macherey and Nagel} at a flow rate of 15 mL/min. The mobile phase was a binary mixture of A (\SI{98.9}{\percent} water, \SI{1}{\percent} acetonitrile, \SI{0.1}{\percent} TFA) and B (\SI{98.9}{\percent} acetonitrile, \SI{1}{\percent} water, \SI{0.1}{\percent} TFA) with a gradient of \SI{3}{\percent} to \SI{45}{\percent} in \SI{30}{\minute}.

UPLC analysis was performed on a \textit{Acquity} UPLC from \textit{Waters} and a BEH300 C18 column (50 x 2.1 mm, \SI{1.7}{\micro\metre} with a flow rate of 0.6 mL/min. A gradient of \SI{3}{\percent} to \SI{40}{\percent} in \SI{6}{\minute} unless otherwise noted. The yield of all isolated peptides were determined via absorption measurements at 280 nm on a \textit{NanoDrop 2000} UV-Vis spectrophotometer.

%-----------------------------------------------------------
%%%%%%%%%%%%%%%%%%%%%%%%%%%%%%%%%%%%%%%%%%%%%%%%%%%%%%%%%%%%

\section{Synthesis of model peptide H-GRAEYSGLG-\ch{NH2} by Fmoc-SPPS}

The model peptide H-GRAEYSGLG-\ch{NH2} \cmpd{cmpd:modelpeptide} was synthesised by standard Fmoc-based solid phase peptide synthesis protocol on an Applied Biosystems Peptide Synthesizer. The peptide was supported on Tentagel Rink amide resin.

%%%%%%%%%%%%%%%%%%%%%%%%%%%%%%%%%%%%%%%%%%%%%%%%%%%%%%%%%%%%

\section{\IUPAC{2-bromo\|-2-(4\|-bromophenyl)\|acetic acid} (\cmpd+{cmpd:bromoPAA})}

\begin{experimental}[delta=(ppm),pos-number=sub,use-equal]
  \sisetup{separate-uncertainty,per-mode=symbol,detect-all,range-phrase=--}

A suspension of \IUPAC{2-(4\-bromophenyl)\|acetic acid} (\SI{10.6}{\gram}, \SI{46.5}{\milli\mol}) and \IUPAC{N-bromo\|succinimide} (\SI{9.932}{\gram}, \SI{55.8}{\milli\mol}) in \ce{CCl4} was irradiated under refluxed with a \SI{240}{\watt} tungsten lamp.

An additional 0.5 eq of NBS was added after \SI{3}{\hour} and the reaction was maintained at reflux for a further \SI{3}{\hour}, at which point the mixture was cooled, filtered, and the concentrated \invacuo. The residual orange solid was purified by flash column chromatography (2:1 cyclohexane:ethyl acetate with \SI{1}{\percent} formic acid). Residual formic acid was removed by co-evaporation with ethyl acetate to yield \cmpd{cmpd:bromoPAA} (\SI{9.733}{\gram}, \SI{33.1}{\milli\mole}, \SI{71}{\percent}) as an off-white crystalline solid.

\NMR(500)[CDCl3] 7.52 (d, J = 8.6 Hz, 2H, ArH), 7.44 (d, J = 8.5 Hz, 2H, ArH), 5.31 (s, 1H, CBrH).
%
\NMR{13,C}(125)[CDCl3] 172.17, 134.24, 132.37, 130.63, 124.14, 44.97.

%
  % \data{MS}[DCP, EI, \SI{60}{\electronvolt}] \val{703} (2, \ch{M+}), \val{582}
  % (1), \val{462} (1), \val{249} (13), \val{120} (41), \val{105} (100).
  % %
  % \data{MS}[\ch{MeOH + H2O + KI}, ESI, \SI{10}{\electronvolt}] \val{720} (100,
  % \ch{M+ + OH-}), \val{368} (\ch{M+ + 2 OH-}).
  % %
  %
\end{experimental}

%%%%%%%%%%%%%%%%%%%%%%%%%%%%%%%%%%%%%%%%%%%%%%%%%%%%%%%%%%%%

\section{\IUPAC{N-methoxy\|-2-\|(4-methoxyphenyl)-\|\|N-\|methyl\|acetamide} (\cmpd+{cmpd:meoamide})}

To a stirred solution of \IUPAC{2-(4-methoxyphenyl)acetic acid} (\SI{1.662}{\gram}, \SI{10}{\milli\mol}) in dry DCM (\SI{50}{\milli\litre}) under argon at \SI{0}{\celsius} was added \IUPAC{N,O-dimethyl\|hydroxylamine hydrochloride} (\SI{0.975}{\gram}, \SI{10}{\milli\mol}) and triethylamine (\SI{1.4}{\milli\litre}, {\SI{10}{\milli\mol}) followed by \IUPAC{1-(3\-dimethyl\|aminopropyl)\-3\-ethyl\|carbo\|diimide hydrochloride} (\SI{2.013}{\gram}, \SI{10.5}{\milli\mol}) in portions over \SI{20}{\minute}. The mixture was maintained at \SI{0}{\celsius} for \SI{15}{\minute} and allowed to rise to room temperature.

After \SI{1}{\hour} the reaction mixture was washed with aq. \ce{HCl} (\SI{1}{\Molar}, 3x\SI{50}{\milli\litre}), sat. \ce{NaHCO3} (3x\SI{50}{\milli\litre}), brine (1x\SI{100}{\milli\litre}), dried over \ce{MgSO4} and the solvent removed \invacuo to yield \cmpd{cmpd:meoamide} (\SI{1.7061}{\gram}, \SI{8.15}{\milli\mole}, \SI{82}{\percent}) as a clear yellow oil which was not further purified.

%%%%%%%%%%%%%%%%%%%%%%%%%%%%%%%%%%%%%%%%%%%%%%%%%%%%%%%%%%%%

\section{\IUPAC{2-bromo-\|2-\|(4-bromophenyl)-\|N-\|methoxy-N-\|methylacetamide} (\cmpd+{cmpd:bromoamide.one})}

To a stirred solution of \cmpd{cmpd:bromoacid} (\SI{4.409}{\gram}, \SI{15}{\milli\mol}) in dry DCM (\SI{120}{\milli\litre}) under argon at \SI{0}{\celsius} was added N,O-dimethylhydroxylamine hydrochloride (\SI{1.463}{\gram}, \SI{15}{\milli\mol}) followed by 1-(3-Dimethylaminopropyl)-3-ethylcarbodiimide hydrochloride (\SI{3.019}{\gram}, \SI{15.75}{\milli\mol}) in portions over \SI{15}{\minute}. Triethylamine (\SI{3.14}{\milli\litre}, \SI{22.5}{\milli\mol}) was then added and the mixture was allowed to rise to room temperature after stirring at \SI{0}{\celsius} for a further \SI{15}{\minute}.

After \SI{1}{\hour} the reaction mixture was washed with aq. \ce{HCl} (\SI{1}{\Molar}, 3x\SI{50}{\milli\litre}), sat. \ce{NaHCO3} (3x\SI{50}{\milli\litre}), brine (1x\SI{50}{\milli\litre}) and dried over \ce{MgSO4}. The solvent was removed \invacuo to yield \cmpd{cmpd:bromoamide.one} (\SI{4.586}{\gram}, \SI{13.6}{\milli\mol}, \SI{91}{\percent} crude yield) as a clear yellow oil which was not further purified.

%%%%%%%%%%%%%%%%%%%%%%%%%%%%%%%%%%%%%%%%%%%%%%%%%%%%%%%%%%%%

\section{\IUPAC{2-bromo-2-\|(4-methoxyphenyl)-\|N-\|methoxy-N-\|methyl\|acetamide} (\cmpd+{cmpd:bromoamide.two})}

To a solution of \cmpd{cmpd:meoamide} (\SI{0.837}{\gram}, \SI{4}{\milli\mol}) in \ce{CCl4} (\SI{20}{\milli\litre}) was added N-bromosuccinimide (\SI{1.068}{\gram}, \SI{6}{\milli\mol}) and azobisisobutyronitrile (\SI{0.067}{\gram}, \SI{0.4}{\milli\mol}). The reaction was stirred overnight at \SI{40}{\celsius}}, washed with water (2x{\SI{20}{\milli\litre}), brine (\SI{20}{\milli\litre}), dried over \ce{MgSO4} and the solvent removed \invacuo to yield \cmpd{cmpd:bromoamide.two} (\SI{0.962}{\gram}, \SI{3.34}{\milli\mol}, \SI{83}{\percent} crude) as a clear yellow oil which was not further purified.

%%%%%%%%%%%%%%%%%%%%%%%%%%%%%%%%%%%%%%%%%%%%%%%%%%%%%%%%%%%%

\section{\IUPAC{S-(1-(4\|-bromophenyl)-2\|-(methoxy\|(methyl)amino)\|-2\|-oxoethyl) O-ethyl carbono\|dithioate} (\cmpd+{cmpd:xanthateamide.one})}

A solution of \cmpd{cmpd:bromoamide.one} (\SI{4.586}{\gram}, \SI{13.6}{\milli\mol}) and potassium ethyl xanthate (\SI{2.180}{\gram}, \SI{13.6}{\milli\mol}) in acetone ({\SI{100}{\milli\litre}}) was stirred at room temperature for \SI{3}{\hour}. The solvent was removed \invacuo to yield a yellow oil which was taken up in water, extracted with \ce{DCM} (2x\SI{50}{\milli\litre}), the organics dried over \ce{MgSO4} and the solvent removed \invacuo to yield a yellow oil which formed a crystalline solid on standing overnight. The solid was purified by column chromatography (4:1 cyclohexane:ethyl acetate) to yield \cmpd{cmpd:xanthateamide.one} as a white crystalline solid (\SI{4.3523}{\gram}, \SI{11.5}{\milli\mol}, \SI{84.5}{\percent}) on standing.

\NMR(500)[CDCl3] 7.47 (d, J = 8.5 Hz, 2H, ArH) 7.35 (d, J = 8.6, 2H, ArH), 6.04 (s, 1H, CHS), 4.61 (q, J = 7.1 Hz, 2H, OCH2), 3.63 (s, 3H, OCH3), 3.21 (s, 3H, NCH3), 1.40 (t, J = 7.1 Hz, 3H, CCH3).

\NMR{13,C}(125)[CDCl3] 132.38, 130.94, 70.77, 61.92, 55.65, 46.24, 33.40, 14.16.

%%%%%%%%%%%%%%%%%%%%%%%%%%%%%%%%%%%%%%%%%%%%%%%%%%%%%%%%%%%%

\section{\IUPAC{S-(1-(4\|-methoxyphenyl)-2\|-(methoxy\|(methyl)amino)\|-2\|-oxoethyl) O-ethyl carbono\|dithioate} (\cmpd+{cmpd:xanthateamide.two})}

To \cmpd{cmpd:bromoamide.two} (\SI{0.962}{\gram}, \SI{3.33}{\milli\mol}) in acetone ({\SI{20}{\milli\litre}}) at \SI{0}{\celsius} were added potassium ethyl xanthate (\SI{0.587}{\gram}, \SI{3.66}{\milli\mol}) in portions over \SI{15}{\minute}. The stirred reaction was allowed to rise to R.T., and after \SI{3}{\hour} the solvent was removed \invacuo to yield a yellow oil. This oil was taken up in DCM, washed with water (2x\SI{50}{\milli\litre}), brine and dried over \ce{MgSO4} to yield a yellow oil which was purified by flash chromatography (4:1 cyclohexane:ethyl acetate) to yield \cmpd{cmpd:xanthateamide.two} (\SI{0.5728}{\gram}, \SI{1.90}{\milli\mol}, \SI{48}{\percent} overall from \cmpd+{cmpd:meoamide}) as a yellow oil.

\NMR(500)[CDCl3] 7.34-7.28 (m, 2H, ArH), 6.82-6.76 (m, 2H, ArH), 5.97 (s, 1H, CHS), 4.54 (q, J = 7.1 Hz, 2H, OCH2), 3.72 (s, 3H, OCH3), 3.52 (s, 3H, NCH3), 3.15 (s, 3H, OCH3), 1.33 (t, J = 7.1 Hz, 3H, CCH3).

\NMR{13,C}(125)[CDCl3] 159.67, 130.08, 114.28, 70.08, 61.44, 55.29, 26.92, 13.77.

%%%%%%%%%%%%%%%%%%%%%%%%%%%%%%%%%%%%%%%%%%%%%%%%%%%%%%%%%%%%
\cmpd*{cmpd:thioamide.one} % Define the bromo first

\section{General Procedure for \cmpd+{cmpd:thioamide}:}
\subsubsection{\IUPAC{2-\|(4-methoxyphenyl)-\|2-\|mercapto-N-\|methoxy-N-\|methylacetamide} (\cmpd+{cmpd:thioamide.two})}

Piperidine (\SI{0.2}{\milli\litre}, \SI{1.93}{\milli\mol}) was added drop-wise to a solution of \cmpd{cmpd:xanthateamide.two} (\SI{0.5757}{\gram}, \SI{1.75}{\milli\mol}) in dry degassed \ce{DCM} under argon on a ice/water bath. An additional 0.4 eq of \ch{piperidine} was added after \SI{1}{\hour}.

\ce{HCl} (\SI{1}{\Molar}, \SI{50}{\milli\litre}) was added after a further hour, the organics separated, washed with \ce{HCl} (\SI{1}{\Molar}, 2x\SI{25}{\milli\litre}), brine, dried over \ce{MgSO4}, and evaporated \invacuo. The residue was purified by flash chromatography (4:1 cyclohexane:ethyl acetate with \SI{1}{\percent} formic acid) to yield \cmpd{cmpd:thioamide.two} (\SI{0.3172}{\gram}, \SI{1.291}{\milli\mol}, \SI{73}{\percent}) as a yellow oil.

\NMR(500)[CDCl3] 7.33-7.25 (m, 2H, ArH), 6.83-6.75 (m, 2H, ArH), 5.09 (d, J = 8.7 Hz, 1H, CHS), 3.72 (s, 3H, OCH3), 3.47 (s, 3H, NCH3), 3.14 (s, 3H, OCH3).

\NMR{13,C}(125)[CDCl3] 159.13, 128.93, 114.15, 109.98, 61.50, 55.31, 42.21, 32.81.

    \subsubsection{\IUPAC{2-\|(4-bromophenyl)-\|2-\|mercapto-N-\|methoxy-N-\|methylacetamide} (\cmpd+{cmpd:thioamide.one})}
    Product \cmpd{cmpd:thioamide.one} obtained as a white crystalline solid (\SI{41}{\percent}).

%%%%%%%%%%%%%%%%%%%%%%%%%%%%%%%%%%%%%%%%%%%%%%%%%%%%%%%%%%%%

\section{General Procedure for \cmpd+{cmpd:DMTamide}:}
\subsubsection{\IUPAC{2-(4\|-bromo\|phenyl)-\|N-methoxy-\|N-methyl-2-\|(4,4'\|-dimethoxy\|trityl\|thio)\|acetamide} (\cmpd+{cmpd:DMTamide.one})}

To a stirred solution of \cmpd{cmpd:thioamide.one} (\SI{1.069}{\gram}, \SI{3.68}{\milli\mol}) in DCM was added \IUPAC{4,4'\|-dimethoxy\|trityl chloride} (\SI{1.372}{\gram}, \SI{4.05}{\milli\mol}) followed by triethylamine (\SI{0.56}{\milli\litre}, \SI{4.05}{\milli\mol}). After \SI{1}{\hour} an additional 0.2 eq of both \ce{4,4'-DMT.Cl} and triethylamine were added. After a further \SI{30}{\minute} the organics were washed with water (2x\SI{70}{\milli\litre}), dried over \ce{MgSO4}, and evaporated \invacuo. The crude was purified by flash column chromatography (4:1 cyclohexane:ethyl acetate + \SI{0.5}{\percent} DMEA) to yield \cmpd{cmpd:DMTamide.one} as a fine white crystalline solid (\SI{1.8419}{\gram}, \SI{3.1}{\milli\mol}, \SI{85}{\percent}).

\NMR(500)[CDCl3] δ 7.34-7.27 (m, 2H, ArH), 7.29-7.04 (m, 8H, ArH), 6.89-6.83 (m, 2H, ArH), 6.80-6.72 (m, 1H, ArH), 6.70-6.63 (m, 4H, ArH), 4.49 (s, 1H, CHS), 3.70 - 3.72 (m, 6H, OCH3), 3.19 (s, 3H, NCH3), 2.96 (s, 3H, OCH3).

\NMR{13,C}(125)[CDCl3] 158.17, 131.34, 131.03, 130.95, 130.11, 129.61, 127.83, 113.02, 77.22, 55.27. The DMT protecting group introduced a complex set of signals which were not readily resolvable.

    \subsubsection{\IUPAC{2-(4\|-methyoxy\|phenyl)-\|N-methoxy-\|N-methyl-2-\|(4,4'\|-\|dimethoxy\|trityl\|thio)\|acetamide} (\cmpd+{cmpd:DMTamide.two})}

    Product \cmpd{cmpd:DMTamide.two} (\SI{0.5517}{\gram}, \SI{1.02}{\milli\mol}, \SI{79}{\percent}) was obtained as a straw yellow crystalline solid.

    \NMR(500)[CDCl3] 7.36-7.27 (m, 2H, ArH), 7.24-7.20 (m, 4H, ArH), 7.18-7.13 (m, 2H, ArH), 7.13-7.07 (m, 2H, ArH), 7.00-6.91 (m, 2H, ArH), 6.82-6.73 (m, 1H, ArH), 6.72-6.62 (m, 6H, ArH), 4.51 (s, 1H, CHS), 3.72-3.70 (m, 6H, OCH3), 3.69 (s, 3H, OCH3), 3.13 (s, 3H, NCH3), 2.91 (s, 3H, OCH3).


%%%%%%%%%%%%%%%%%%%%%%%%%%%%%%%%%%%%%%%%%%%%%%%%%%%%%%%%%%%%

\section{General Procedure for \cmpd+{cmpd:DMTaldehyde}}

\subsubsection{\IUPAC{2-(4\|-bromophenyl)\|-2-(\|4,4'\|-dimethoxy\|trityl\|thio)\|acetaldehyde} (\cmpd+{cmpd:DMTaldehyde.one})}

A solution of \ce{LiAlH4} (\SI{3}{\Molar} in \ce{THF}, \SI{220}{\micro\litre}) was added drop-wise over the course of \SI{10}{\minute} to a stirred solution of the amide \cmpd{cmpd:bromoDMTamide} (\SI{1.068}{\gram}, \SI{3.68}{\milli\mol}) in dry THF (\SI{30}{\milli\litre}) under argon while cooling in a dry ice/\ce{isopropanol} bath.

The reaction was quenched after \SI{30}{\minute} by addition of \ce{NaHSO4} {\SI{1}{\Molar}, \SI{30}{\milli\litre}) drop-wise at first, followed by DCM (\SI{100}{\milli\litre}) and water (\SI{50}{\milli\litre}. The organics were separated, washed with \ce{NaHSO4} solution (\SI{1}{\Molar}, \SI{50}{\milli\litre}) and the combined aqueous washes were back extracted with DCM (\SI{20}{\milli\litre}). The combined organics were dried over \ce{MgSO4} and the solvent evaporated \invacuo to yield a yellow oil which was purified by flash chromatography (4:1 cyclohexane:ethyl acetate with \SI{0.5}{\percent} DMEA) to yield the protected auxiliary \cmpd{cmpd:DMTaldehyde.one} (\SI{0.2112}{\gram}, \SI{0.40}{\milli\mol}, \SI{61}{\percent}) as a red crystalline solid.

    \subsubsection{\IUPAC{2-(4\|-methoxyphenyl)\|-2-(\|4,4'\|-dimethoxy\|trityl\|thio)\|acetaldehyde} (\cmpd+{cmpd:DMTaldehyde.two})}
    Protected auxiliary \cmpd{cmpd:DMTaldehyde.two} (\SI{0.1884}{\gram}, \SI{0.389}{\milli\mol}, \SI{78}{\percent}) was obtained as a red crystalline solid.

%%%%%%%%%%%%%%%%%%%%%%%%%%%%%%%%%%%%%%%%%%%%%%%%%%%%%%%%%%%%

\section{General Procedure: Introduction of auxiliary onto model peptide \cmpd+{cmpd:modelpeptide} (\cmpd+{cmpd:auxiliarypeptide})}

To auxiliary \cmpd{cmpd:DMTaldehyde} (15 eq) in \SI{350}{\micro\litre} of an NMP:i-PrOH:AcOH (4:1 and \SI{5}{\percent}) solution was added \ch{NaBH3CN} (15 eq). This reductive amination mixture was shaken with peptide \cmpd{cmpd:modelpeptide} (\SI{7}{\micro\mol}) on solid support for \SI{6}{\hour}. The peptide was subsequently treated with the cleavage mixture (95:5 TFA:TIS; \SI{2}{\milli\litre}) for \SI{2}{\hour}, precipitated from cold \ch{Et2O} and centrifuged at 20000 RPM for \SI{20}{\minute}. The crude was purified by preparative RP-HPLC to give the auxiliary-peptide \cmpd{cmpd:auxiliarypeptide}.

\textbf{(Br-Aux)-GRAEYSGLG-\ce{NH2} (\cmpd{cmpd:auxiliarypeptide.one}):}, observed mass 1124.45 Da, calculated (M+H)\textsuperscript{+} 1122.4 Da, \textepsilon\textsubscript{280} 1961 M\textsuperscript{-1} cm\textsuperscript{-1}, \SI{7.77}{\percent} yield.

\textbf{(MeO-Aux)-GRAEYSGLG-\ce{NH2} (\cmpd{cmpd:auxiliarypeptide.two}):}, observed mass 1074.56 Da, calculated (M+H)\textsuperscript{+} 1074.5 Da, \textepsilon\textsubscript{280} 2450 M\textsuperscript{-1} cm\textsuperscript{-1}, \SI{24.4}{\percent} yield.

%%%%%%%%%%%%%%%%%%%%%%%%%%%%%%%%%%%%%%%%%%%%%%%%%%%%%%%%%%%%

\section{General Procedure: Peptide Ligation}

Auxiliary peptide \cmpd{cmpd:auxiliarypeptide} (1 eq) and peptide thioester \cmpd{cmpd:thioester} (1 eq) were dissolved in degassed ligation buffer (\SI{10}{\percent} \ch{CH3CN}, \SI{100}{\milli\Molar} TCEP, \SI{100}{\milli\Molar} \IUPAC{Na2HPO4}, \SI{3}{\percent} thiophenol) which had been adjusted to pH 7.5 with \SI{2}{\Molar} \ch{NaOH}. The reaction mixture was shaken at room temperature until UPLC analysis indicated complete consumption of the auxiliary peptide. The auxiliary dipetide was purified by preparative RP-HPLC.

\textbf{LYRAG-(Br-Aux)-GRAEYSGLG-\ce{NH2} (\cmpd{cmpd:auxiliarydipeptide.one}):}, observed mass 842.97 Da, calculated (M+2H)\textsuperscript{2+}/2 841.9 Da, \textepsilon\textsubscript{280} 3241 M\textsuperscript{-1} cm\textsuperscript{-1},
 \SI{83}{\percent} yield.

\textbf{LYRAG-(MeO-Aux)-GRAEYSGLG-\ce{NH2} (\cmpd{cmpd:auxiliarydipeptide.two}):} observed mass 817.98 Da, calculated (M+2H)\textsuperscript{2+}/2 817.9 Da, \textepsilon\textsubscript{280} 3730 M\textsuperscript{-1} cm\textsuperscript{-1}, \SI{83}{\percent} yield.

%%%%%%%%%%%%%%%%%%%%%%%%%%%%%%%%%%%%%%%%%%%%%%%%%%%%%%%%%%%%

% ------------------------------------------------------------------------

\section{General Procedure: Auxiliary Cleavage}

A solution of the ligated auxiliary peptide \cmpd{cmpd:auxiliarydipeptide} (\SI{30}{\nano\mole}) in degassed aqueous cleavage solution (20-400 mM TCEP, 80-1600 mM morpholine, \SI{60}{\micro\litre}) was shaken in a \SI{200}{\micro\litre} Eppendorf tube at the experimental temperature for \SI{24}{\hour}. The reaction progress was followed by UPLC analysis. Conversion to the native peptide was evaluated by calculating the relative area of the native peptide peak to the areas of the native peptide and peptide side product peaks. The differing molar absorption coefficients for the products at 280 nm were taken into account.

\textbf{H-LYRAGGRAEYSGLG-\ce{NH2} (\cmpd{cmpd:dipeptide}):} The formation of the ligated native peptide was observed after auxiliary cleavage from both \cmpd{cmpd:auxiliarydipeptide.one} and \cmpd{cmpd:auxiliary.dipeptide.two}. \textepsilon\textsubscript{280} 2560 M\textsuperscript{-1} cm\textsuperscript{-1}.
% ------------------------------------------------------------------------

%%% Local Variables:
%%% mode: latex
%%% TeX-master: "../thesis"
%%% End:


% \backmatter % book mode only
\appendix
\chapter{Appendix}

\ifpdf
    \graphicspath{{Appendix/AppendixFigs/PNG/}{Appendix/AppendixFigs/PDF/}{Appendix/AppendixFigs/}}
\else
    \graphicspath{{Appendix/AppendixFigs/EPS/}{Appendix/AppendixFigs/}}
\fi

  \begin{figure}[!htpb]
      \includegraphics[width=\textwidth]{MeoBrAuxPeps.png}
      \caption{UPLC trace of 4-bromophenyl (\cmpd+{cmpd:auxiliarypeptide.one}) and 4-methoxyphenyl (\cmpd+{cmpd:auxiliarypeptide.two}) auxiliary peptides}
      \label{fig:auxpepuplc}
  \end{figure}

\begin{figure}
    \includegraphics[max width=\linewidth]{br50cleavage.png}
    \caption{UPLC trace for the cleavage of \cmpd+{cmpd:auxiliarydipeptide.one} (100 mM TCEP, \SI{50}{\celsius})}
    \label{fig:br50cleave100}
\end{figure}

\begin{figure}
    \includegraphics[max width=\linewidth]{brrtcleavage.png}
    \caption{UPLC trace for the cleavage of \cmpd+{cmpd:auxiliarydipeptide.one} (400 mM TCEP, R.T.)}
    \label{fig:brrtcleavage}
\end{figure}

\begin{figure}
    \includegraphics[max width=\linewidth]{meoinitiatorcleavage.png}
    \caption{Side product formation during cleavage of \cmpd+{cmpd:auxiliarydipeptide.two} in the presence of a radical initator (VA-044 20 eq.)}
    \label{fig:meoinitatorcleavage}
\end{figure}

\begin{figure}
    \includegraphics[max width=\linewidth]{meocleavage.png}
    \caption{Rate of auxiliary peptide \cmpd+{cmpd:auxiliarydipeptide.two} cleavage}
    \label{fig:meocleavagerate}
\end{figure}
% ------------------------------------------------------------------------

%%% Local Variables:
%%% mode: latex
%%% TeX-master: "../thesis"
%%% End:


\bibliographystyle{plainnat}
%\bibliographystyle{Classes/CUEDbiblio}
%\bibliographystyle{Classes/jmb}
%\bibliographystyle{Classes/jmb} % bibliography style
\renewcommand{\bibname}{References} % changes default name Bibliography to References
\bibliography{References/references} % References file

\end{document}
