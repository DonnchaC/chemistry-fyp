\def\baselinestretch{1}
\chapter{Conclusions and Further Work}
\ifpdf
    \graphicspath{{Conclusions/ConclusionsFigs/PNG/}{Conclusions/ConclusionsFigs/PDF/}{Conclusions/ConclusionsFigs/}}
\else
    \graphicspath{{Conclusions/ConclusionsFigs/EPS/}{Conclusions/ConclusionsFigs/}}
\fi

\def\baselinestretch{1.66}

In this project we successfully synthesized two phenyl substituted ligation auxiliaries via a six-step route from commercially available starting materials. We then demonstrated that both compounds can effectively facilitate auxiliary-mediated ligation at a Gly-Gly site in a model peptide system. Finally we showed that both auxiliaries can be cleaved rapidly under mild conditions. In doing so we synthesized a 14-mer cysteine-free peptide. Our auxiliary system can be removed under neutral conditions which is a major improvement on the HF and TFA conditions required for the removal of the previously reported auxiliaries \cmpd{cmpd:methoxymercaptoethyl} and \cmpd{cmpd:trimethoxymercapto}.

Future work could focus on tuning the auxiliary substituent groups in an effort to enhance the rate and selectiveness of the cleavage step. It would be illuminating to investigate the effect of substituents with a large -I effect as they should exert a stabilizing effect on the proposed benzylic radical intermediate.

The characteristic isotope mass spectra of bromine may allow for the identification of auxiliary cleavage fragments from the cleavage of auxiliary \cmpd{cmpd:auxiliarypeptide.one}. A more complete understanding of the cleavage mechanism will help focus further research on this class of ligation auxiliaries.

%%% ----------------------------------------------------------------------

% ------------------------------------------------------------------------

%%% Local Variables:
%%% mode: latex
%%% TeX-master: "../thesis
%%% End:
