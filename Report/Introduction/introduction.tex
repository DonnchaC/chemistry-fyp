%%% Thesis Introduction --------------------------------------------------
\chapter{Introduction}
\ifpdf
    \graphicspath{{Introduction/IntroductionFigs/PNG/}{Introduction/IntroductionFigs/PDF/}{Introduction/IntroductionFigs/}}
\else
    \graphicspath{{Introduction/IntroductionFigs/EPS/}{Introduction/IntroductionFigs/}}
\fi

Importance of proteins and petides. The ability to synthetically produce peptides provides fantastic tools to probe biological process and functions by cemical means.

The development of solid phase peptide synthesis provided the key technique for the sequential sythesis of proteins. Unfortunatly SPPS encouters difficult  when producing longer peptides due to aggregation factors. The sequential loses during a linear synthesis also lead to poor yields in longer sequences or in peptides with so called 'difficult sequeneces'.

\section{Peptide Ligation}

early approaches

\section{Native Chemical Ligation}

Capture followed by S->A acyl shift. First reported by '...'

Peptide ligation techniques help to overcome some of shortcommings of solid phase peptide synthesis. Allows for convergent synthesis of longer chain peptiddes or proteins by ligating small segments which are producable via SPPS.

Particularly useful for synthesis of long, and cyclic peptides
Allows synthesis of challenging sequences not possible by SPPS

site some examples of successes from NCL.
extended native chemical ligation

\subsection{Limitations of Native Chemical Ligation}

The requirment for an N-terminal cystine in standard Native Chemical Ligation is limitation. Cysteine has relativily low abundance in proteins (ref).

In some cases it is possible to substitue a cysteine for a (glycine/alanine) to produce a non-native peptide which maintains its natural function.

\section{ Native Chemical Ligation at Non-Cysteine Sites}

A number of approaches developed to extend the utility of NCL

    \subsection{Desulfurization Approaches}

    Synthesis of peptides conta

    \subsection{Auxiliary Mediated NCL}


    scheme for auxiliary mediated ligation

    Limited range of possible junctions,
    Issues with auxiliary removal

    Previously developed auxiliaries.
    3,4,5-trimethoxy and others.
    table of their yields/cleavage

    photolabile auxilaries

    current work focuses on a new set of auxiliaries

    2-phenyl-2-mercapto auxiliaries. This class of auxiliries have shown to be removable in the presence of TCEP. Previous work has indicated the auxiliary removals occurs by way of a radical mechanism.

    In this work two ring substituted 2-phenyl-2-mercapto auxiliaries are synthesised to probe robe substituent effects on both ligation efficieny but primarily on radical promoted auxiliary cleavage.


        \subsubsection{Auxillary Removal}

        The mechanism of removal for this class of auxiliaries is currently unknown. This work studies a the cleave of a 2-(4-bromophenyl)-2-mercapto system. It is hoped that presence of bromine in the auxiliary should allow for the identificasistion of auxiliary cleavage products from the post cleavage crude by mass spectrum/MALDI analysis.

        The synthesising removal for this cl bromoauxiliary was choosen to probe the relative influence of inductive and resonance effects on the ring and the stabilisation of the hypothesised radical intermediate

        site, TCEP radical sulfhydral cleavage.

%%% ----------------------------------------------------------------------


%%% Local Variables:
%%% mode: latex
%%% TeX-master: "../thesis"
%%% End:
